\documentclass[svgnames]{article}
\usepackage{pgfgantt}
\usepackage{xcolor}
\usepackage{tcolorbox}
\usepackage{sectsty}
\usepackage{titlesec}
\usepackage{chronology}
\usepackage{fontawesome}
\usepackage{academicons}
\usepackage{hyperref}
\usepackage{tabularx}
\usepackage{marvosym}
\usepackage{ifsym}
\usepackage[portrait, margin=2.5cm]{geometry}
\usepackage{tikz}
\usepackage{colortbl}
\usepackage{natbib}
\urlstyle{rm}

% Colors
\definecolor{CVblue}{HTML}{1483D3}
\definecolor{CVmain}{HTML}{1483D3}
\definecolor{CVsecondary}{HTML}{6BBAA7}
\definecolor{CVter}{HTML}{FFE77A}

\definecolor{CVgrey}{HTML}{E6E6E6}
\sectionfont{\color{CVmain}}
\subsectionfont{\color{white}}

\newcommand*\circled[1]{\tikz[baseline=(char.base)]{
            \node[shape=circle,draw,inner sep=1pt,color=CVsecondary] (char) {#1};}}

% Fonts
\usepackage{fontspec}

\setmainfont{Lato}[
    Path=/home/ctroupin/.fonts/Lato2OFL/,
    Extension = .ttf,
    UprightFont=*-Regular,
    BoldFont=*-Bold,
    ItalicFont=*-Italic,
    BoldItalicFont=*-BoldItalic
    ]
    
%\newfontfamily\thesubsectionfont{/home/ctroupin/.fonts/D-DIN.ttf}
\newfontfamily\thesubsectionfont[Path = /home/ctroupin/.fonts/,Color=CVsecondary]{D-DIN}
\titleformat*{\subsection}{\large\thesubsectionfont}
%\urlstyle{rm}

\hypersetup{bookmarksopen=true,
bookmarksnumbered=true,  
pdffitwindow=false, 
pdfstartview=FitH,
pdftoolbar=false,
pdfmenubar=false,
pdfwindowui=true,
pdfauthor=Charles Troupin,
pdftitle=Charles Troupin Curriculum,
pdfsubject=C. Troupin Resume,
colorlinks=true,%
breaklinks=true,%
linkcolor=CVmain,anchorcolor=CVmain,%
citecolor=CVmain,filecolor=blue,%
menucolor=CVmain,%
urlcolor=CVmain}

\newcommand{\sepa}{$\cdot$~}
\newcommand{\role}[1]{\textbf{#1}}


\newcolumntype{P}[1]{>{\centering\arraybackslash}p{#1}}

\parskip .2cm

\begin{document}
\pagestyle{empty}
\thispagestyle{empty}

%\begin{table}
%\centering
%\begin{tabular}{P{.05\textwidth}P{.8\textwidth}P{.05\textwidth}}
% & Charles \textbf{Troupin} & \\
% & Engineer in Physics, Master in Oceanography \& PhD in Science (Oceanography) & \\
% \hline
%
%\end{tabular}
%\end{table}

\hfill Liège, \today

\vspace{.5cm}

Dear Dr. Lips,

\vspace{.5cm}

As a researcher at the University of Liège (Belgium), specialised in oceanography and data management, I contact you to apply to the position of "Science Officer". I have been in contact with EuroGOOS since the beginning of my career, through meetings and partnerships in projects, and seen what is has achieved. This motivates me for the present application. 

Since 2006, I have been involved in a variety of European projects and initiatives, related to different aspect of the ocean: SeaDataNet~I, II, SeaDataCloud, CMEMS In Situ Thematic Assembly Center, ODIP, EMODnet Physics, Biology, Chemistry, \ldots 
Thanks to these projects, I am aware of the peculiarities of ocean data management, for example the importance of metadata or the quality control. Working in such projects also makes me conscious of the diversity of European projects, services or initiatives and the necessity to try to foster convergence between them. They also helped me build a robust professional network across Europe. 

Between 2006 and 2011, I worked on a thesis entitled "\textit{Study of the Cape Ghir upwelling filament using variational data analysis and regional numerical model}". The thesis used both numerical modelling (ROMS model) and \textit{in situ} data analysis to study a complex coastal area off Morocco. In the frame of this research, I participated in 2009 to a 3-week campaign in this area; it contributed to make me understand the real value of \textit{in situ} observations and the inherent difficulty to collect them.

Later in my career, from March 2014 to January 2017, I worked as the head of the Data Center at the SOCIB (Spain). This role involved the management of the complete data workflow, continuous interactions with the other facilities (gliders, HF radar, modelling, \ldots) and the participation to various European initiatives related to data management. Finally, this position required strong organization skills for the management of deliverable, the writing of reports, as well as the human resources management (job description, candidate interview and selection).

Since 2017, I am working as a researcher at the University of Liège, where I focus my attention on data analysis technique (spatial interpolation) applied to different types of data: temperature, salinity, velocities from high-frequency radar systems.

In addition to technical and scientific aspects of the work, I also organised conferences and training courses. In 2016, I co-organised the 48 edition of the Liège Colloquium on Ocean Dynamics, which gathered more than 200 participants from 25 countries around the world. In the frame of SeaDataNet, I organised training workshops on DIVA, a software tool developed in our group and used to create climatologies using \textit{in situ} data. 

Making available the largest quantity of data to the community has always been a goal: at SOCIB, the Data Center closely worked with the MedClic team (\url{https://medclic.es/en/}) in order to make easier the data access to non-scientist users, while also contributing to the elaboration of educational resources. I believe that the main objectives presented in the EuroGOOS 2030 Strategy are in line with my conception of the operational oceanography, in particular I think sustainability is a challenge to be addressed. 

Overall, I am confident that my experience in ocean-related topics, combined with the network acquired during these years, will be an excellent match for this job. This is why I would love the chance to further discuss the position.

\vspace{.5cm}
Thank you for your consideration.

\vspace{1.cm}
\hfill Charles \textsc{Troupin}

%\newpage
%
%\nocite{*}
%\bibliographystyle{copernicus.bst}
%
%\bibliography{Troupin_publi.bib}
%



\end{document}






- Proven project management experience;

→ project management at my current position, but more relevant at SOCIB as the head of the Data Center. Managing the communitation and traning activities of the CMEMS INSTAC

- Fluent in spoken and written English, knowledge of French will be an advantage;

→ languages: French mothertongue; Fluent in English and Spanish; intermediate level in Catalan, basic reading understanding of German, Italian and Dutch

- Experience in delivering presentations at meetings and conferences.

→ add list of presentations; frequently receive good feedback from colleagues after a talk

Desirable competencies
- Experience of seagoing data collection or observing systems;


- Knowledge of oceanographic models or products for maritime users;


- Experience as a partner or researcher in EU-funded projects;


- Experience in communications and outreach;


- Experience in stakeholder engagement.


Inter-personal competencies


- Highly organised, responsible, and improvement-minded;


- Ability to work independently and as part of a team;


- Task-orientated and resourceful;


- Quality and detail-oriented;


- Transparency in working and a team-orientated work ethic;


- Ability to multi-task and prioritise;


- Ability to deliver on allocated tasks and respond in a timely manner to deadlines;


- Strong social and presentation skills.


\end{document}


CV 
Letter of motivation in English
Dr. Inga Lips, EuroGOOS
inga.lips@eurogoos.eu.
Deadline for applications: 9 February 2022


Strategy 2030
-------------

General EuroGOOS
----------------

founded 1994
44 members
18 countries

Goals: data sharing , operational oceanography, technological dev.

5 ROOS + working groups + task teams

focus to climate and ocean health applications to respond to
the needs of the European society

- best practices
- unlock marine data
- advance modelling and forecasting capabilities
- deliver common strategies, priorities, and standards for
an integrated, sustained, and fit-for-purpose European
ocean observing.

Objectives
----------

1 STIMULATE communities of practice
2 ADVOCATE for coordinated and integrated European ocean observing and operational oceanography.
3 STRENGTHEN and expand partnerships
4 PROMOTE sustainability across the value chain of operational oceanography and ocean observing
5 MOBILISE the public on the importance of the ocean and oceanographic services


