Running in Barcelona

On continue dans la série "Running in ...", cette fois-ci à Barcelone. De ce que j'ai pu voir, la plupart des joggueurs courent le long de la côte - le port, la Barceloneta etc - et évidemment ce n'est pas trop dans le style de la maison.

[Barcelona2018_0470.JPG

Lors d'un autre trip dans la capitale de la Catalogne, j'avais pu expérimenter des parcours sur les hauteurs de la ville, avec d'incroyables vues, mais pour ce faire il était nécessaire d'avoir un hôtel situé dans la zone, sinon ça voulait dire 4 à 5 km avant d'arriver dans la partie haute.

Cette fois-ci, hôtel oblige, je me suis dis que l'exploration du Parc de Montjuïc pouvait être une bonne option. Ce n'était pas la première fois que j'y mettais les pieds, mais cette fois-ci j'ai pu voir que je n'en connaissais pas tous les recoins. Par exemple le chateau située au somment de la coline... aucune idée! 

Barcelona2018_0443.JPG

Il y a de très nombreux chemins, soit de type sentier, soit plutôt aslphalte, et par conséquent il est aisé de combiner différentes montées et descentes en fonction des besoins d'entrainemets. Bonne surprise au lever du soleil: la silhouette des sommets de Mallorca, visible depuis où j'étais. Cela se produit plusieurs fois l'année, mais d'habitude les observations se font d'un peu plus haut (Tibidabo par exemple).

[bcn_mallorca.jpg]

Pour revenir au thème principal: j'ai aussi testé les longues, très longues avenues que comptent la ville. On peut faire plus de 3 km en ligne droite, et si le traffic le permet, sans s'arrêter. J'avoue que c'est assez chiant à la longue, même si le changement de quartiers rend le parcours moins monotone.

[mapTracksBCN]

Toujours sur ma liste: monter sur le Tibidabo, point que l'on peut voir de n'importe où en ville. Ça a l'air très faisable mais comme toujours l'obstacle numéro un s'appelle le temps.


